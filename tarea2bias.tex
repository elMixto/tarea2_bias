%% LyX 2.3.6.1 created this file.  For more info, see http://www.lyx.org/.
%% Do not edit unless you really know what you are doing.
\documentclass[english]{article}
\usepackage[T1]{fontenc}
\usepackage[latin9]{inputenc}
\usepackage[active]{srcltx}
\usepackage{babel}
\usepackage{graphicx}
\usepackage[unicode=true]
 {hyperref}

\makeatletter

%%%%%%%%%%%%%%%%%%%%%%%%%%%%%% LyX specific LaTeX commands.
%% Because html converters don't know tabularnewline
\providecommand{\tabularnewline}{\\}

\makeatother

\begin{document}
\title{Tarea 2: Bias lies and causal inference: data interpretation in the
ubiquitous}
\author{Jos� Gonz�lez Cort�s}

\maketitle
Todo el c�digo con generado para la obtenci�n de datos se puede encontrar
en el repositorio: \href{https://github.com/elMixto/tarea2_bias}{github.com/elMixto/tarea2\_bias}

\section*{Part A: How common is your name?}

Esta pregunta est� abierta a 2 interpretaciones, desde mi punto de
vista:

\paragraph{�Cuan com�n es que se registre tu nombre a un beb� actualmente?}

Se obtiene midiendo directamente la proporcion asociada al nombre
``Jos�'' para el a�o 2021 en el conjunto de datos.

\begin{center}

\includegraphics[scale=0.6]{jose_over_time}

\end{center}

El uso del nombre ha decrecido de forma lineal desde un m�ximo de
\textbf{4.8\%} Jos� registrados en 1920 hasta un \textbf{0.5\% }en
2021 que es su popularidad en la actualidad.

\paragraph{�Cuan com�n es tu nombre entre los chilenos?}

Para esta pregunta se considera a todos los chilens registrados en
la base de datos, vivos o muertos aquellos que se han registrado como
``Jos�'' en algun momento.

El numero total de nombres registrados es de \textbf{22,129,100} y
el n�mero total de ``Jos�'' registrados es de \textbf{588,613.}

En base a esto, la probabilidad de elegir aleatoriamente a un chileno
registrado en esta bas� de datos, y que se llame Jos� es de un \textbf{2.65\%.}

\section*{Part B: Did the beatles impact the registration of chilean baby names}

Se conocen los nombres los integrantes de la banda:
\begin{itemize}
\item John Lennon
\item Paul McCartey
\item George Harrison
\item Ringo Starr
\end{itemize}
Adem�s todos tienen g�nero masculino, se puede obtener una evoluci�n
historica de sus nombres atrav�z del tiempo:

\begin{center}

\includegraphics[scale=0.6]{individual_beatles}

\end{center}

Se puede apreciar un notable aumento en el n�mero de registros por
a�o desde 1960 hacia adelante, adem�s se nota que los primeros nombres
de los integrantes son los �nicos en notarse a esta escala, por lo
que se puede despreciar la influencia de sus apellidos.

Para comprobar que este efecto es solo para estos nombres, se usa
un contraste con nombres comunes usados en Reino Unido, y se comprueba
sus registros historicos.

\begin{center}

\includegraphics[scale=0.6]{beatles_constrat}

\end{center}

En base a estas observaciones se puede afirmar con cierta seguridad
que, la formaci�n de la banda, si gener� una tendencia para el registro
de nombres con los integrantes de la banda.

\section*{Part C: What can you say about the gender distribution in Chile across
the years? Do you see any trends and/or singularities?}

El an�lisis de distribucion g�nero se puede realizar de forma bastante
sencilla simplemente obteniendo las proporciones totales por cada
a�o.

La distribuci�n de registro de nombres se ha mantenido estable, a
excepcion de una anomal�a en el a�o 1920 con una alta discrepancia
entre los g�neros, con 42.5\% de hombres y 57.5\% de mujeres en ese
a�o, de esta forma si se consideran los rangos desde 1920 hasta 2021
y desde 1921 hasta 2021 se obtienen las siguientes caracteristicas
de la distribucion de proporciones de hombres, mujeres e indeterminado.

\begin{center}

\begin{tabular}{|c|c|c|c|c|c|}
\hline 
\textbf{Sexo (1920-2004)} & \textbf{M} & \textbf{F} & \textbf{Sexo (1921-2004)} & \textbf{M} & \textbf{F}\tabularnewline
\hline 
\hline 
Min & 42.50\% & 48.60\% & Min & 48.59\% & 48.60\%\tabularnewline
\hline 
Max & 51.39\% & 57.49\% & Max & 51.39\% & 51.41\%\tabularnewline
\hline 
Average & 50.31\% & 49.69\% & Average & 50.40\% & 49.59\%\tabularnewline
\hline 
std & 1.03 & 1.03 & std & 0.59 & 0.59\tabularnewline
\hline 
\end{tabular}

\end{center}

Si se comparan las variables de ambos datos hasta 2004 excluyendo
el a�o 1920 el promedio cambia muy poco pero la desviacion estandar
se reduce en un 42.7\%.

\begin{center}

\begin{tabular}{|c|c|c|c|}
\hline 
\textbf{Sexo (2005-2021)} & \textbf{M} & \textbf{F} & \textbf{I}\tabularnewline
\hline 
\hline 
Min & 50.81\% & 48.65\% & 0.004\%\tabularnewline
\hline 
Max & 51.34\% & 49.18\% & 0.120\%\tabularnewline
\hline 
Average & 51.04\% & 48.94 & 0.084\%\tabularnewline
\hline 
std & 0.125 & 0.1206 & 0.002\tabularnewline
\hline 
\end{tabular}

\end{center}

En el a�o 2005 comienzan a aparecer beb�s registrados con sexo indeterminado
de forma que es necesario decidir si tiene significancia para la evaluacion,
en este caso el numero de instancias en que se registrar valores para
I es muy peque�o variando entre 10 y 28 personas.

(insertar gr�fico)

\begin{center}

\begin{tabular}{|c|c|c|}
\hline 
\textbf{Sexo (1921-2021)} & \textbf{M} & \textbf{F}\tabularnewline
\hline 
\hline 
Min & 48.59\% & 48.60\%\tabularnewline
\hline 
Max & 51.39\% & 51.41\%\tabularnewline
\hline 
Average & 50.40\% & 49.59\%\tabularnewline
\hline 
std & 0.59 & 0.59\tabularnewline
\hline 
\end{tabular}

\end{center}

De esta forma considero la mejor representacion estad�stica de los
datos, despues de limpiarlos, eliminando el a�o 1920, con una desviacion
estandar muy peque�a de 0.59, por lo que la distribucion de g�nero
es b�sicamente constante. Lo que se evidencia en el siguiente gr�fico.

\begin{center}

\includegraphics[scale=0.6]{gender_distribution}

\end{center}
\end{document}
